\chapter{Course Introduction}

\section{Course Overview and Objectives}

The "Computing Infrastructures" course at Politecnico di Milano, led by Professors Manuel Roveri, Gianluca Palermo, and Danilo Ardagna, offers a comprehensive introduction to modern, large-scale datacenters. The core principle of the course is understanding that modern datacenters necessitate the seamless integration of diverse components, including applications, computation nodes, storage devices, and networks, into a unified computing infrastructure. The curriculum is designed to cover the fundamentals of current datacenter architectures, ranging from the analysis of individual components to the entire global infrastructure.

\section{Key Topics Covered}

The course is structured into three primary areas of study:

\begin{itemize}
    \item \textbf{Hardware (HW) Infrastructures:} This section explores the physical components of datacenters.
    \begin{itemize}
        \item \textit{System-level:} Focuses on Computing Infrastructures, Data Center Architectures, and Rack/Structure design.
        \item \textit{Node-level:} Covers Servers (computation, HW accelerators), Storage (Type, technology), and Networking (architecture and technology).
        \item \textit{Building-level:} Addresses essential support systems like cooling, power supply, and failure recovery mechanisms.
    \end{itemize}
    
    \item \textbf{Software (SW) Infrastructures:} This area delves into the software that manages and orchestrates the hardware.
    \begin{itemize}
        \item \textit{Virtualization:} Includes Process/System VMs and Virtualization Mechanisms like Hypervisors (Para/Full virtualization).
        \item \textit{Computing Architectures:} Explores Cloud Computing (types, characteristics), X-as-a-service models, and Edge/Fog Computing.
        \item \textit{Machine and deep learning-as-a-service:} Discusses the delivery of ML/DL capabilities as a service.
    \end{itemize}
    
    \item \textbf{Methods:} This part of the course equips students with analytical tools and theoretical knowledge.
    \begin{itemize}
        \item \textit{Reliability and availability of datacenters:} Covers definitions, fundamental laws, and RBDs.
        \item \textit{Disk performance:} Analyzes different disk types, performance metrics, and RAID configurations.
        \item \textit{Scalability and performance of datacenters:} Introduces definitions, fundamental laws, and queuing network theory.
    \end{itemize}
\end{itemize}

\section{Lecturers and Administration}

The course is delivered by a team of faculty, each with a specific area of expertise:
\begin{itemize}
    \item \textbf{Prof. Danilo Ardagna:} HW-SW infrastructure, performance.
    \item \textbf{Prof. Manuel Roveri:} Disks technologies, ML as a Service.
    \item \textbf{Roberto Sala:} Dependability.
    \item \textbf{Prof. Marco Gribaudo:} Disks dependability and performance exercises.
\end{itemize}

All course materials, including lecture slides, will be published on the official WeBeep website for the course: \textbf{095898 - COMPUTING INFRASTRUCTURES (ARDAGNA DANILO) [2024-25]}. While lectures are not streamed live, recordings will be made available on the platform during the weekend.

\section{Evaluation and Assessment}

The student evaluation will be based on a final written, closed-book exam. The exam will consist of:
\begin{itemize}
    \item A set of exercises and simple problems to solve, similar to those that will be covered during the classes.
    \item Questions (True/False and open) dealing with more general topics covered by the course.
\end{itemize}
Additionally, students have the opportunity to earn up to \textbf{2 bonus points} by participating in four quizzes held during the classes (0.5 points per quiz). These bonus points are valid \textbf{only for the first June call}.

\section{Bibliography and Resources}

The course recommends several key texts, for which PDFs are freely available:
\begin{itemize}
    \item \textit{The Datacenter as a Computer: Designing Warehouse-Scale Machines, 3rd Edition (2018)} by Luiz André Barroso and Urs Hölzle.
    \item \textit{Quantitative System Performance: Computer System Analysis Using Queueing Network Models} by Edward D. Lazowska, John Zahorjan, G. Scott Graham, and Kenneth C. Sevcik.
    \item \textit{Fault Tolerance - Reliable Systems from Unreliable Components} by Jerome H. Saltzer \& M. Frans Kaashoek.
\end{itemize}
Additional reading materials and resources will be provided to students via the WeBeep platform.
